\chapter*{Enunciado}

\begin{enumerate}
      \item
            Escribir un programa \textit{hello world} en C, que haga parpadear el led conectado en la placa pro mini. Puede utilizar el código del apunte \textit{El primer programa embebido} aunque este puede contener errores (esta advertido).

            Utilice el esquemático de la placa para reconocer cuál es el pin en el microcontrolador \textbf{AVR} al que esta conectado el led.

            \textbf{Recursos}
            \begin{itemize}
                  \item Utilice el makefile que se provee con este TP en el source code de ejemplo.
                  \item Clase del primer programa embebido
            \end{itemize}

            Responda:

            \begin{enumerate}[label=\arabic*]
                  \item ¿A que pin de la placa pro mini está conectado el led?
                  \item A que pin del microcontrolador \textbf{AVR} esta conectado el led?
            \end{enumerate}

      \item
            Crear una segunda aplicación para \textbf{AVR} que controle 5 pines de E/S digital paralela del atmega328p (GPIO).

            \vspace{0.3cm}

            El hardware a utilizar son 5 \textbf{LEDs} para \textbf{SALIDA}. Todo el hardware debe ser controlado por pines conectados al periférico de E/S \textit{GPIO PORTB}.

            \vspace{0.3cm}

            \textbf{Software: } Escriba un programa que realice el efecto del \textit{knight-rider}, utilizando los 5 leds.

      \item
            Utilice \textbf{avr-objdump} para realizar un decodificado del programa binario elf (obtener el código en lenguaje ensamblador AVR a partir del binario elf). Observe ahora que al compilar el programa para un microcontrolador particular aparece al principio del código la tabla de vectores.

            \vspace{0.3cm}

            Presente en este ejercicio las primeras 10 posiciones de la tabla de vectores.

      \item
            Responder:
            \begin{enumerate}
                  \item ¿Qué hw existe en cada dispositivo de E/S que permite su programación?
                  \item ¿Cómo se llaman en el PORT B del AVR?
                  \item ¿Cuantos dispositivos de E/S observa conectados al DATABUS del AVR? Mencione sus nombres
                  \item Usted debe controlar un nuevo dispositivo de E/S en un sistema, utilizando un programa escrito en C. ¿Que detalles necesita conocer para escribir el software? ¿De dónde obtiene los detalles?
                  \item ¿Qué es el voltaje? ¿Qué es la corriente?
                  \item ¿Por qué se lo clasifica como \textit{arquitectura harvard} a los micros AVR de 8-bits?
            \end{enumerate}
\end{enumerate}

\textbf{Verificación: } Compilar, vincular y enviar el firmware a cada AVR.

\vspace{0.3cm}

\textbf{Entrega: } Subir (push) el trabajo práctico resuelto (o sus versiones intermedias) al repositorio git compartido.