\chapter*{Enunciado}

\begin{enumerate}
      \item
            Obtener el código fuente del tp3 y analizar sus archivos.

            \begin{enumerate}
                  \item Leer los archivos fuentes del práctico. Entender el archivo cabecera \textit{serial.h} y observar como es utilizado por \textit{main.c}.
                  \item Agregue un archivo \textit{Makefile} al proyecto (puede utilizar el del \textbf{tp2} modificando los nombres de los archívos del proyectos). Verifique el \textit{Makefile} con \textbf{Make clean; Make}.
            \end{enumerate}

      \item Desarrollar un driver (controlador) para el periférico UART del atmega328p, utilizando los archivos propuestos.

            \begin{enumerate}
                  \item Comience completando la estructura de datos que hace \textit{overlay} con los registro del hardware del USART del atmega328p.

                        \begin{itemize}
                              \item Lea y comprenda utilizando la hoja de datos la ubicación de los registros (página 612 del manual del atmega328p).
                              \item Estudie minimamente la descripción de cada registro del periférico (página 191 del manual del atmega328p).
                        \end{itemize}
                  \item Lea nuevamente el código fuente de \textit{main.c} para comprender la manera en que main utiliza la API del driver.
                  \item Escriba la rutina de inicialización. Utilice un baud rate de $9.600$ bits por segundo, 8 bits de datos, sin bit de paridad, y un bit de stop. Utilice el puntero a la estructura de registros para configurar el UART y activar también la recepción y transmición.
                  \item Escriba las rutinas \textit{serial\_put\_char()} y \textit{serial\_get\_char()}, utilizando E/S programada.
            \end{enumerate}

      \item Utilice \textit{cutecom} o \textit{minicom} tal vez como root o con sudo, depende del sistema linux) para comunicarse con el microcontrolador avr. El dispositivo serial en Linux será del estilo \textbf{/dev/ttyUSBX} (puede utilizar el comando dmesg cuando conecta el arduino pro mini con el adapdator USB, para conocer el dispositivo correcto), y utilice los mismo parametros de comunicación que la aplicación embebida.

      \item Ampliar el código fuente para que \textit{main} espere un byte desde la PC:

            \begin{itemize}
                  \item Si la PC envió la letra \textit{c} el sistema realiza el parpadeo del led en PB5. (Utilice el código del TP anterior).
                  \item Si la PC envió la letra \textit{k} el sistema realiza el efecto \textbf{Knight Rider}
            \end{itemize}
\end{enumerate}